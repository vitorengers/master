% =============================================================================
% Chapter 1: Introduction
% =============================================================================
% Master Thesis on Radio Occultation with CubeSats
% -----------------------------------------------------------------------------

\chapter{INTRODUCTION}

The observation and prediction of atmospheric conditions are fundamental to a wide range of societal activities, from weather forecasting and aviation safety to climate monitoring and disaster preparedness. Among the remote sensing techniques that have emerged over the past three decades, \textbf{radio occultation} (RO) has established itself as a uniquely valuable method for profiling the Earth's atmosphere with high vertical resolution, global coverage, and long-term stability \cite{Kursinski1997GPSRO, Anthes2011ROReview}.

Radio occultation exploits the refraction of radio signals from Global Navigation Satellite Systems (GNSS), such as GPS, Galileo, and GLONASS, as these signals traverse the atmosphere on their way to receivers aboard Low Earth Orbit (LEO) satellites. The bending and delay of the signals encode information about the atmospheric refractive index, which can be inverted to retrieve vertical profiles of temperature, pressure, humidity, and electron density \cite{Hajj2002TechnicalDescription}. Unlike optical and infrared sensors, GNSS-RO measurements are unaffected by clouds and precipitation, providing reliable observations under all weather conditions.

The technique has proven particularly impactful for numerical weather prediction (NWP), where RO observations have been identified as one of the most cost-effective data sources per observation for improving forecast accuracy, especially in data-sparse regions such as the Southern Hemisphere oceans and polar areas \cite{Cucurull2007Impact, Poli2010NWPImpact}. Furthermore, the inherent calibration stability of GNSS signals---traceable to atomic frequency standards---makes RO data an essential component of climate monitoring systems \cite{Steiner2013Validation}.

% -----------------------------------------------------------------------------
\section{THE CUBESAT REVOLUTION IN ATMOSPHERIC OBSERVATION}
% -----------------------------------------------------------------------------

Traditionally, GNSS-RO missions have relied on large, expensive satellite platforms developed by governmental space agencies. Missions such as CHAMP (2000), GRACE (2002), and COSMIC/FORMOSAT-3 (2006) demonstrated the scientific and operational value of RO, but their high costs and long development cycles limited the number of deployed receivers and, consequently, the spatial and temporal density of observations \cite{Anthes2008COSMIC, Wickert2001CHAMP}.

The emergence of \textbf{CubeSat technology} has fundamentally transformed this landscape. CubeSats are miniaturized satellites built to standardized form factors (1U $\approx$ 10~cm $\times$ 10~cm $\times$ 10~cm), enabling rapid development, low-cost launch as secondary payloads, and deployment of large constellations \cite{Harnisch2023CubeSatRO}. Commercial operators such as \textbf{Spire Global} and \textbf{GeoOptics} now operate CubeSat constellations that collectively provide tens of thousands of RO profiles per day, rivalling or exceeding the output of traditional government missions \cite{Bowler2020SpireNWP, Mannucci2012CICERO}.

This democratization of access to RO observations creates new opportunities but also introduces significant engineering challenges. CubeSat platforms impose strict constraints on size, weight, and power (SWaP), which directly affect the design of GNSS receivers and onboard data processing systems. The limited downlink bandwidth available to small satellites often necessitates onboard data reduction, filtering, or even complete retrieval processing before transmission to ground stations \cite{Furano2020AISpace}.

% -----------------------------------------------------------------------------
\section{MOTIVATION: EMBEDDED ALGORITHMS FOR CUBESAT RO}
% -----------------------------------------------------------------------------

A key challenge in CubeSat-based radio occultation is the development of \textbf{efficient prediction and estimation algorithms} capable of operating within the SWaP envelope of small satellite platforms. These algorithms consume pre-processed RO data---such as bending angle profiles or refractivity retrievals---and produce application-specific outputs, including:

\begin{itemize}
    \item Atmospheric profiles (temperature, pressure, humidity) at specified vertical levels;
    \item Derived indices for weather or climate applications;
    \item Quality flags and uncertainty estimates for data assimilation.
\end{itemize}

The implementation of such algorithms on embedded hardware presents a compelling trade-off between \textbf{computational fidelity} and \textbf{resource consumption}. Two broad classes of embedded platforms are commonly considered for space applications:

\begin{enumerate}
    \item \textbf{Microcontrollers and Microprocessors (MCU/CPU):} These offer programming flexibility and ease of development but may be limited in throughput for computationally intensive tasks.
    
    \item \textbf{Field-Programmable Gate Arrays (FPGAs):} These reconfigurable devices can achieve high parallelism and deterministic timing, making them attractive for real-time signal processing, but they require specialized development workflows and careful management of fixed-point arithmetic \cite{Bjelogrlic2020FPGASpace}.
\end{enumerate}

The choice between these platforms---or hybrid architectures combining both---depends on factors such as algorithm complexity, latency requirements, power budget, and radiation tolerance. A systematic comparison of implementations across different embedded platforms is essential for informing the design of future CubeSat RO missions.

% -----------------------------------------------------------------------------
\section{OBJECTIVES}
% -----------------------------------------------------------------------------

\subsection{General Objective}

The general objective of this dissertation is to \textbf{develop, implement, and compare prediction algorithms for radio occultation data on embedded platforms}, specifically contrasting FPGA-based and microcontroller-based solutions in terms of performance, resource utilization, and energy efficiency.

\subsection{Specific Objectives}

\begin{itemize}
    \item Conduct a comprehensive review of the state of the art in radio occultation methodology, with emphasis on CubeSat missions and onboard processing requirements.
    
    \item Define a prediction algorithm that consumes pre-processed RO data and produces atmospheric estimates suitable for near-real-time applications.
    
    \item Implement the algorithm on an FPGA platform, employing fixed-point arithmetic and high-level synthesis (HLS) techniques.
    
    \item Implement the same algorithm on a representative microcontroller or microprocessor platform.
    
    \item Develop a benchmarking framework to evaluate and compare the two implementations in terms of:
    \begin{itemize}
        \item Prediction accuracy (RMSE, bias, correlation);
        \item Computational latency and throughput;
        \item Resource utilization (memory, logic cells, DSP blocks);
        \item Power and energy consumption.
    \end{itemize}
    
    \item Provide design guidelines for selecting embedded platforms for CubeSat RO processing based on mission requirements.
\end{itemize}

% TODO(project): Specify the exact algorithm to be implemented.
% TODO(project): Specify the target FPGA and MCU platforms (e.g., Xilinx Zynq, STM32).
% TODO(project): Define the input data source (real RO profiles or simulated data).

% -----------------------------------------------------------------------------
\section{CONTRIBUTIONS}
% -----------------------------------------------------------------------------

This work contributes to the field of satellite remote sensing and embedded systems by:

\begin{enumerate}
    \item Providing a detailed methodology for adapting radio occultation retrieval algorithms to resource-constrained embedded platforms.
    
    \item Presenting a novel comparative analysis of FPGA and microcontroller implementations for atmospheric prediction, with quantitative benchmarks.
    
    \item Offering practical insights and design recommendations applicable to the next generation of CubeSat RO missions.
\end{enumerate}

% -----------------------------------------------------------------------------
\section{DOCUMENT STRUCTURE}
% -----------------------------------------------------------------------------

This dissertation is organized as follows:

\begin{itemize}
    \item \textbf{Chapter 2 -- Methodology:} Presents the scientific foundations of radio occultation, including physical principles, measurement geometry, signal processing, inversion methods, and the role of CubeSat platforms. This chapter also describes the data sources and quality control procedures relevant to this study.
    
    \item \textbf{Chapter 3 -- Prediction Algorithm:} Defines the prediction algorithm used in this work, including problem formulation, mathematical framework, and computational complexity analysis.
    
    \item \textbf{Chapter 4 -- Hardware Implementation:} Describes the implementation of the algorithm on FPGA and microcontroller platforms, including design choices, optimization strategies, and development tools.
    
    \item \textbf{Chapter 5 -- Results and Discussion:} Presents the experimental results, including benchmark comparisons and analysis of trade-offs between the platforms.
    
    \item \textbf{Chapter 6 -- Conclusions:} Summarizes the main findings, discusses limitations, and suggests directions for future work.
\end{itemize}

% TODO(project): Adjust chapter titles and descriptions as the thesis structure evolves.
