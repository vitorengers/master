% ---
% RESUMO
% ---

% resumo em português
\setlength{\absparsep}{18pt} % ajusta o espaçamento dos parágrafos do resumo
\begin{resumo}
 Esta dissertação investiga métodos para medir a impedância térmica em módulos IGBT, com foco na análise de confiabilidade, identificação de falhas e desempenho térmico. São discutidos os principais modos de falha, os tipos de encapsulamento e a relevância dos ensaios de vida acelerada na avaliação de módulos IGBT. Simulações baseadas na variação dos parâmetros do modelo de Foster foram realizadas para representar a degradação do encapsulamento, avaliando seus impactos na temperatura virtual de junção, na impedância térmica e na função estrutural cumulativa. Os resultados ajudam a identificar áreas críticas de potencial falha. No âmbito experimental, foram realizadas medições da temperatura virtual de junção e comparações entre os métodos estático e dinâmico para a determinação da impedância térmica, ambos apresentando resultados consistentes. O estudo enfatiza a necessidade de considerar a não linearidade da condutância térmica em função da temperatura, fundamental para análises precisas. Adicionalmente, são comparados dois métodos de medição da impedância térmica entre a junção e o case em módulos de potência sem baseplate: o método estático, que monitora a curva de resfriamento após a aplicação de um pulso prolongado de potência, e o método dinâmico, baseado na aplicação sucessiva de pulsos de potência de diferentes larguras.

 \textbf{Palavras-chave}: Caracterização de Impedância Térmica; Confiabilidade de Módulos IGBT; Medição de Temperatura Virtual de Junção; Análise de Função Estrutural Cumulativa; Método de Medição Estático; Método de Medição Dinâmico.
\end{resumo}
