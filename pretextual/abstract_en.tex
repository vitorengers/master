% ---
% ABSTRACT
% ---

% resumo em inglês
\begin{resumo}[Abstract]
 \begin{otherlanguage*}{english}
   This dissertation investigates methods for measuring thermal impedance in IGBT modules, focusing on reliability analysis, fault identification, and thermal performance. The main failure modes, encapsulation types, and the relevance of accelerated lifetime testing for IGBT modules are discussed. Simulations based on variations in Foster model parameters were conducted to represent encapsulation degradation, assessing its impacts on virtual junction temperature, thermal impedance, and cumulative structural function. The results contribute to identifying critical areas of potential failure. On the experimental side, virtual junction temperature measurements were carried out, along with comparisons between static and dynamic methods for determining thermal impedance, both yielding consistent results. The study highlights the importance of accounting for the nonlinearity of thermal conductance as a function of temperature, which is essential for accurate analyses. Additionally, two methods for measuring thermal impedance between the junction and the case in baseplate-less power modules are compared: the static method, which monitors the cooling curve after applying a prolonged power pulse, and the dynamic method, based on successive power pulses of varying widths.

   \textbf{Keywords}: Thermal Impedance Characterization; Reliability of IGBT Modules; Virtual Junction Temperature Measurement; Cumulative Structure Function Analysis; Static Measurement Method; Dynamic Measurement Method.
 \end{otherlanguage*}
\end{resumo}
